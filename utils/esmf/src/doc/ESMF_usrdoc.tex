% $Id: ESMF_usrdoc.tex,v 1.1.6.1 2002/04/24 03:26:02 erik Exp $

\documentclass[]{article}

\usepackage{epsf}
\usepackage{html}
\usepackage[T1]{fontenc}

\textwidth 6.5in
\textheight 8.5in
\addtolength{\oddsidemargin}{-.75in}

\begin{document}

\bodytext{BGCOLOR=white LINK=#083194 VLINK=#21004A}

\begin{titlepage}

\begin{center}
{\Large Earth System Modeling Framework } \\
\vspace{.25in}
{\Large {\bf Earth System Modeling Framework User's Guide}} \\
\vspace{.25in}
{\large {\it David Neckels, Cecelia DeLuca, others}}
\vspace{.5in}
\end{center}

\begin{latexonly}
\vspace{5.5in}
\begin{tabular}{p{5in}p{.9in}}
\hrulefill \\
\noindent {\bf NASA High Performance Computing and Communications Program} \\
\noindent Earth and Space Sciences Project \\
\noindent CAN 00-OES-01 \\
\noindent http://www.esmf.ucar.edu \\
\end{tabular}
\end{latexonly}

\end{titlepage}

\tableofcontents

\newpage

\section{Introduction}

This Guide to the Earth System Modeling Framework (ESMF) is intended 
for a new user.  It outlines the scope, structure, and function of the 
ESMF software in order to help the user understand what benefits the ESMF 
can provide.  It also describes how to install the ESMF software and how 
to incorporate it into applications.

The ESMF Software Developer's Guide contains detailed information on how 
to extend the ESMF system.  Reference manuals are available for individual 
libraries within ESMF; see these for additional information on interfaces 
and usage.

%\section{Description}
% $Id: ESMF_desc.tex,v 1.1.6.1 2002/04/24 03:26:01 erik Exp $

\section{Description}

This section will be filled in as the ESMF effort matures.


%\section{Installation}
% $Id: ESMF_install.tex,v 1.1.6.1 2002/04/24 03:26:01 erik Exp $

\section{Installation}

Currently the following environment variables need to be set:
\begin{verbatim}
  ESMF_DIR      top-level ESMF directory
  ESMF_ARCH     platform and compiler configuration
\end{verbatim}

\noindent The following configurations are supported:

\begin{tabular}{lll}
{\tt ESMF\_ARCH}  & {\tt alpha}      &  OSF1, Native compilers. \\
                  & {\tt IRIX64}     &  IRIX64, MIPSpro/mpt 64 bit compilers. \\
                  & {\tt IRIX}       &  IRIX64, MIPSpro/mpt n32 abi compilers. \\
                  & {\tt rs6000\_sp}  &  AIX, mpxlf90\_r, mpcc\_r, and mpCC\_r.  \\
                  & {\tt linux\_gnupgf90} & Linux, pgf90, gcc and g++.  \\
                  & {\tt linux\_pgi}  &  Linux, pgf90, pgcc, pgCC. \\
                  & {\tt linux\_lf95} &  Linux, lf95, gcc, g++. \\
                  & {\tt solaris}        &  SunOs, SUNWspro compilers. \\
                  & {\tt solaris\_hpc}   &  SunOs, SUNWhpc compilers. \\
\end{tabular}

\smallskip

The library requires {\tt gmake} to build.  Simultaneous multiple architecture builds are supported, with
one restriction; the test cases may only be run on one platform at a time. 

\smallskip

\noindent Build the library with the command:
\begin{verbatim}
  gmake BOPT=g  
\end{verbatim}
  for a debug version or
\begin{verbatim}
  gmake BOPT=O  
\end{verbatim}
  for an optimized version.

A test suite is included with the library.  Tests are provided for both MPI
and uniprocessor builds. 

\noindent To build and run MPI C tests:

\begin{verbatim}
  gmake BOPT=g test_c
\end{verbatim}

\noindent To build and run MPI F90 tests:
\begin{verbatim}
  gmake BOPT=g test_f90
\end{verbatim}

\noindent To build and run non-MPI C tests:
\begin{verbatim}
  gmake BOPT=g test_cuni
\end{verbatim}

\noindent To build and run non-MPI F90 tests:
\begin{verbatim}
  gmake BOPT=g test_f90uni
\end{verbatim}

Output files from the test examples will be directed to files in:
\begin{verbatim}
${ESMF_DIR}/test${BOPT}/${ESMF_ARCH}
\end{verbatim}

Exhaustive tests exist that may be activated by setting the environment variable
{\tt ESMF\_EXHTEST} to {\tt on}.  While this activation will test the library more thoroughly,
it will take significantly longer to complete than the basic tests.

\smallskip

\noindent To build documentation:
\begin{verbatim}
  gmake dvi           ! Makes the dvi files
  gmake pdf           ! Makes the pdf files
  gmake html          ! Creates the html directory.
  gmake alldoc        ! Builds all the above documents.
\end{verbatim}

To use the library from C/C++, link with the library executable and include
the {\tt ``ESMC.h''} file.
To use the library from Fortran, link with the library executable and
create links to the library modules in your build directory.  These are
in the top level {\tt mod} directory under the appropriate architecture.  Alternately, 
most compilers have a module-include-path directive which may be used to point
to the correct module directory.
To include the library in application modules, {\tt USE} the
module, e.g. {\tt ESMF\_TimeMgmtMod}.  

There is an install target which will copy the library and mod files to an
install location.  To invoke this target use:
\begin{verbatim}
  gmake BOPT=[O,g] ESMF_LIB_INSTALL=dir_for_lib ESMF_MOD_INSTALL=dir_for_mod_files install 
\end{verbatim}

Some users may wish for the library to be built in a directory different from 
where the source code resides.  To do this, build using:
\begin{verbatim}
   gmake ESMF_BUILD=build_directory_here BOPT=[O,g]
\end{verbatim}

The {\tt ESMF\_BUILD} variable gives an alternate path in which to place the libraries,
mod files and object files.  This variable defaults to {\tt ESMF\_DIR}.  If it is 
assigned another value, the {\tt ESMF\_BUILD} variable will need to be passed as
an additional argument to the the above make commands.  (Alternatively the variable
{\tt ESMF\_BUILD} can be set in the environment (using setenv or export) and then it 
need not be passed to any make calls).



%\section{Basic Usage and Conventions}
% $Id: ESMF_usage.tex,v 1.1.6.1 2002/04/24 03:26:02 erik Exp $

\section{Basic Usage and Conventions}

\subsection{Bindings}

The library includes both C/C++ and F90 bindings.  The C/C++ prefix for
procedures and parameters is {\tt ESMC\_} and the F90 prefix is {\tt ESMF\_}.

\subsection{Object Function Conventions}

A large portion of the library is built using objects.  To standardize the
creation, use and destruction of these objects the library uses certain
common function for each object.  

\begin{itemize}
\item{\bf ESM[F/C]\_<Class>New}
Allocates a {\it deep} object.  This function will allocate space from the heap and may
create other resources which must be free'd or deallocated.  All items created with
{\bf New} must be deleted with {\bf Delete} below.  {\tt e.g. ESMF\_LogNew}

\item{\bf ESM[F/C]\_<Class>Delete}
De-Allocates an object created with New.  Any transient resources that were created will
be cleaned up. {\tt e.g. ESMF\_LogDelete}

\item{\bf ESM[F/C]\_<Class>Init}
Initializes a {\it shallow} object.  This class of objects does not need to be destructed and
is guaranteed not to allocate any resources that must be cleaned up.  {\tt e.g. ESMF\_TimeInit}

\item{\bf ESM[F/C]\_<Class>Set<Value>}
Sets a given value with the class.  The {\bf Value} parameter is decided by the class. 
{\tt e.g. ESMF\_LogSetState}

\item{\bf ESM[F/C]\_<Class>Get<Value>}
Gets a given value with the class.  The {\bf Value} parameter is decided by the class.  
{\tt e.g. ESMF\_LogGetState}

\item{\bf ESM[F/C]\_<Class>SetConfig}
This function takes a list of resorces as defined in the resource section the class.  Some class
may not have resources and this function has no meaning.  The function allows a user to set multiple
resources with one function call.  {\tt e.g. ESMF\_LogSetConfig}

\item{\bf ESM[F/C]\_<Class>GetConfig}
This function takes a list of resorces as defined in the resource section the class.  Some class
may not have resources and this function has no meaning.  The function allows a user to get multiple
resources with one function call.  {\tt ESMF\_LogGetConfig}

\item{\bf ESM[C]\_<Class>Construct}
This function fills initializes an object with valid data.  This function is called by both the
{\bf Init} and {\bf New} functions.  Depending on the type of object this function may or may not 
allocate resources that need to be freed.

\item{\bf ESM[C]\_<Class>Destruct}
This function cleans up any resources that were created in the {\bf Construct} method.

\end{itemize}

\subsection{Constraints}

The library design imposes some constraints on the user:

\begin{itemize}
\item {\it No direct access of Fortran derived types.}  Attributes
of Fortran derived types are private.

\item{\it All types should be initialized.} In order to provide consistent
argument checking and to increase the overall robustness of the library,
a user should call one of the {\tt ``Init''} routines before using the
library's Fortran derived types, or the {\tt ``Construct''} routine before 
using its C/C++ classes.  A {\tt ``New''} routine is provided for C/C++ if
dynamic memory allocation is desired.

\end{itemize}

\subsection{Error Handling}

All C/C++ procedures return an integer error code.  All F90 procedures have 
an optional integer return code argument (with the exception of a select few
functions that use {\tt stdargs}).  Return codes are translated 
into error descriptions using the methods: 

\begin{verbatim}
    void ESMC_ErrPrint(int rc)

    subroutine ESMF_ErrPrint(rc)  
    integer, intent(in), optional :: rc
\end{verbatim}

A return code of {\tt ESM[F/C]\_SUCCESS} indicates that an 
operation executed without errors.

The user can currently choose from two different error handlers.
{\tt ESM[F/C]\_ERR\_RETURN} will simply return from a routine in which an error 
is identified, without printing an error description.
{\tt ESM[F/C]\_ERR\_EXIT} will print a detailed error description including
file, function name, and line number, and will then terminate execution
(this is a simple exit, not an {\tt MPI\_ABORT}).  These handlers are set 
using the methods: 
\begin{verbatim}
    void _ErrHandlerSetType(ESM[F/C]_ErrHandlerType type)

    subroutine ESM[F/C]_ErrHandlerSetType(type)
    integer, intent(in) :: type
\end{verbatim}

The intent is to provide an error handling system in which
users can choose from a variety of handlers or supply their own.








\end{document}
