% $Header: /fs/cgd/csm/models/CVS.REPOS/shared/esmf/src/Infrastructure/TimeMgmt/doc/Attic/TimeMgmt_reqdoc.tex,v 1.1.6.1 2002/04/24 03:25:49 erik Exp $
% $Name: cam2_0_1_brnchT_release3 $

\documentclass[english]{article}
% \usepackage[T1]{fontenc}
% \usepackage[latin1]{inputenc}
\usepackage{babel}
\usepackage{theorem}
\usepackage{fancyhdr}
\usepackage{lastpage}
\usepackage{supertabular}
\usepackage{html}

\newcommand{\abbr}{TMG }
\newcommand{\docversion}{Draft1}

\pagestyle{fancy}
\lhead{\abbr Requirements: \docversion }
\chead{}
\rhead{}
\lfoot{\abbr Requirements: \docversion }
\cfoot{}
\rfoot{\thepage\ of \pageref{LastPage}}

%%

\newtheorem{requirement}{\Large \abbr REQ}
\newtheorem{subreq}{\abbr REQ}[requirement]

\newenvironment
{reqlist}
{\begin{list} {} {} \rm \item[]}
{\end{list}}

\newcommand{\req}[1]{
\begin{requirement}
\addcontentsline{toc}{section}{R\arabic{requirement}{\it\bf\ #1}}
{\Large #1}
\end{requirement}
}

\newcommand{\sreq}[1]{
\begin{subreq}
\addcontentsline{toc}{section}{  R\arabic{requirement}.\arabic{subreq}{\it\ #1}}
{#1}
\end{subreq}
}
%%



\makeatletter
\makeatother

\begin{document}

\title{ESMF Time Manager Requirements Document}


\author{Brian Eaton, Chris Hill}
\date{}

\maketitle

\tableofcontents

\newpage
\part*{Part I - Overview}
\addcontentsline{toc}{section}{Part I - Overview}


\section{Introduction}

This document lays out the detailed requirements for the ESMF Time Manager (\abbr). 
The \abbr requirements document is the mechanism by which features that will be 
included in the \abbr implementation are identified. The \abbr implementation 
will contain features that derive from the requirements laid out in Part II of the 
\abbr requirements document (this document) or from requirements laid out in 
subsequent revisions of the \abbr requirements document or from the general 
ESMF coding standards and practices documents.

\section{Document Structure}

This document contains a brief background section summarizing the overall role 
of the \abbr. The terms section lists key terms that are used in the \abbr 
requirements. The \abbr requirements section gives an itemized list of detailed 
requirements. Each requirement item is numbered using a two level numbering 
scheme. The primary number refers to a {\it broad requirement} and the 
secondary number refers to {\it sub-requirements}. The {\it sub-requirements} 
clarify and qualify the broad requirement they relate to. Background information
is given after each requirement.


\section{\abbr Background}

Many of the components that will run and interact within the ESMF
are prognostic simulation codes employing time-stepping approaches
to solve a numerical implementation of a set of mathematical equations.
Coordinating component to component interactions and coordinating interactions
between components and external systems, from which data is ingested
or to which data is exported, requires precise notions of time. The
role of the \abbr is to provide a standard set of \char`\"{}time services\char`\"{}
to components running under ESMF. The \abbr set of \char`\"{}time services\char`\"{}
will be part of the ESMF Infrastructure layer. Including these services
withing the ESMF Infrastructure layer permits

\begin{itemize}
\item Development of multiple components with compatible notions of time.
\item Development of a robust, core library of common time functions.
\end{itemize}

The \abbr is not intended to be a comprehensive set of services for
all time related operations. For example it will not include a complete
time-zone capability that supports translation between all of the
more than 300 time-zones used on the Earth today. However, the \abbr
will provide a generic foundation for developing libraries that do
provide bespoke time services. As such the \abbr is anticipated to
be a library of software that is targeted at both ESMF component 
developers and at specialized library developers.

\section{\abbr Terms}

\tablefirsthead{}
\tablehead{\Large \bf \abbr Terms cont...\vspace{2mm} \\}
\tabletail{}
\tablelasttail{}

\begin{supertabular}{ll}
{\Large \bf alarm}&
\multicolumn{1}{p{95mm}}{
The \abbr implementation will provide a concept that supports behaviors 
like the behavior of the alarm described in Part II. An alarm is an event 
that occurs at a particular time (or set of times).  It is like a real 
alarm on a real alarm clock except that, in order to determine whether 
it is "ringing", an alarm is "read" by an explicit application action.


}\\
{\Large \bf clock}&
\multicolumn{1}{p{95mm}}{
The \abbr implementation will provide a concept that supports behaviors like 
the behavior of the clock described in Part II. A clock in this document can 
be thought of as having many of the attributes of a real clock. However, most 
clocks used in ESMF components have a key difference to a real clock. Clocks 
in an ESMF component are generally stepped forward by the component, as an 
explicitly coded step within the overall component. The discussion of an 
external clock requirement is the only exception to this behavior.

}\\

{\Large \bf priority}&
\multicolumn{1}{p{95mm}}{
Requirements are given a priority in order to help the ESMF development team 
and software engineering manager schedule development work. The priority 
categories used in the requirements documents are

\vspace{1mm}

{\bf Essential Priority:} This refers to a requirement that must be 
implemented for the software to be of any practical use.

\vspace{1mm}

{\bf Very Useful Priority:} This refers to a requirement that is needed by 
many components. If a very useful requirement is not implemented, then potentially 
several groups will need to develop significant software of their own to meet 
the requirement.

\vspace{1mm}

{\bf Useful Priority:} This refers to a requirement that is needed by one or more
of the components in the ESMF test suite. If a useful priority requirement is not 
implemented, then a group will need to develop or adapt some software of their own to 
meet the requirement.

\vspace{1mm}

{\bf Optional Priority:} This refers to a requirement that is seen as valuable, but that
is either not currently employed by existing applications or that can be
implemented by application and library developers provided the appropriate
base \abbr features allow for it. If time permits, optional priority 
requirements will be implemented. Optional priority requirements are also factored into 
implementation choices for higher priority requirements. Including optional priority
requirements ensures that, as far as possible, implemented features will be compatible 
with later addition of features to support the optional requirements.
}\\

\vspace{2mm}
{}&{}\\
\vspace{2mm}

{\Large \bf \abbr} &
\multicolumn{1}{p{95mm}}{Standard abbreviation for the ESMF Time Manager.
}\\

\vspace{2mm}

{\Large \bf time instant}&
\multicolumn{1}{p{95mm}}{Generic name for an absolute time and
date specification. A time instant is made up of a time and date and an 
associated calendar and time-zone. ``Jan 3rd 1999, 03.30:24.56s, UTC'' is one 
example of a time instant. Another example of a time instant is 36.3719 seconds 
in the default calendar and time zone.
}\\

\vspace{2mm}

{\Large \bf time interval}&
\multicolumn{1}{p{95mm}}{Generic name for a time and date period. Examples of 
time intervals are 36.3719 seconds, 17 hours, 8 years and 3 days 4 hours.
The precise meaning of units and hours and years is dependent on the calendar
associated with the time interval.
}
\end{supertabular}

\newpage
\part*{Part II - \abbr Requirements}
\addcontentsline{toc}{section}{Part II - \abbr Requirements}

\req{Multiple clock and alarm support is required.}
\sreq{Components should be able to create and manipulate multiple clocks and alarms.}
\begin{reqlist}
{\bf Background:} In many numerical approaches several different time-steps are 
used for different elements of the system. Multiple clocks are important to keep
track of these different time-steps. Ensemble simulations may also require
members to proceed with different temporal trajectories, all within a single 
component.\\
{\bf Originators:} All participants.\\
{\bf Priority:} Essential.
\end{reqlist}
\sreq{Components should be able to delete their clocks and alarms.}
\begin{reqlist}
{\bf Background:}
Sometimes temporary clocks and alarms will be needed. A mechanism to both create and
destroy alarms and clocks on the fly is required.
\\
{\bf Originators:} 
All participants.
\\
{\bf Priority:} Very Useful.
\end{reqlist}
\sreq{Component should be able to get a list of their clocks and alarms.}
\begin{reqlist}
{\bf Background:}
A useful bookkeeping feature would be the ability to find out what different
clocks a component owns at any time.
\\
{\bf Originators:} 
All participants.
\\
{\bf Priority:} Useful.
\end{reqlist}

\req{Cross-component clock and alarm queries and manipulations are required.}
\begin{reqlist}
\end{reqlist}

\sreq{Components should be able to query the clock(s) or alarms of another
        component and components should be able to manipulate certain clock(s) 
        and alarms of another component.}
\begin{reqlist}
{\bf Background:}
Synchronized shutdown notifications will be sent to components through clock
and alarm settings. These notifications may be generated by a high-level control program
or a coupler component. However, components should be able to determine 
which clocks and alarms can be externally manipulated so that they can assume that
certain clocks and alarms are private.

{\bf Originators:} 
All participants.
\\
{\bf Priority:} Essential.
\end{reqlist}

\sreq{Components should be able to "label" their clocks and alarms.}
\begin{reqlist}
{\bf Background:}
When coordinating notions of time among components with several
clocks or alarms, it is important to be able to identify the role of 
each clock or alarm.
For instance, a component might want to declare a master clock that 
defines the overall time within a component, or components
might need to declare a standard alarm that can be used
by a higher-level driver layer or a coupler component to trigger 
the generation of restart information or to trigger a clean shutdown.
\\
{\bf Originators:} 
All participants.
\\
{\bf Priority:} Essential.
\end{reqlist}


\req{Multiple, widely used, schemes for specifying time instants 
and time intervals are required.}
\begin{reqlist}
{\bf Background:}
Different applications have differing needs for how time is represented.
Some idealized simulations may not require a notion of a true calendar based
time instant. Time
relative to the start of the simulation is the relevant quantity
for these applications. Nevertheless, standard time units such as milli-seconds, 
seconds, minutes, hours and days are useful to these applications. However, the notion 
of a formal calendar date
is not relevant to these applications. Other applications require precise
notions of calendar date relative to the current Gregorian calendar (with a 365.2425 days 
average "year" and appropriate leaping) or relative
to "standard" calendar conventions (for example a 360 day, 12$\times$30 day month year)
for paleo-simulation. In some applications data ingestion cycles
need to be precisely coordinated with an actual time instant as specified by standard 
military and civil clock resources. In these cases leap seconds, which adjust navigational 
clocks to accommodate ongoing variations in the Earth angular momentum budget
(the difference between Coordinate Universal Time (UTC) and International
Atomic Time (TAI)), are required.

In principle, facilities for representing units could be layered on top
of a TMG foundation layer that uses a single universal and uniform
resolution clock, for example seconds and nano-seconds since the start of the universe 
(this is similar to the TAI time that is maintained). However, calendar based 
representations of time instants and time intervals are ubiquitous
requirements of ESMF components. The TMG should therefore support
a broad span of widely used schemes for specifying time. The supported schemes that
are to be supported are listed below.
\\
{\bf Originators:} 
All participants.
\\
{\bf Priority:} Very Useful.
\end{reqlist}

\sreq{Components should be able to use a flexible {\bf YY MM DD H M S MS O} form for specifying
time instants and time intervals.}
\begin{reqlist}
{\bf Background:}
For both a time instant and a time interval {\bf YY} is a year number,  
{\bf MM} is a month number, {\bf DD} is a day number, {\bf H} is an hour number,
{\bf M} is a minute number, {\bf S} is a second number, {\bf MS} is a millisecond number
 and O is a time-zone offset (specified
in whole hours and minutes) number. The valid ranges of these numbers for time instants 
depend on the calendar in use. Valid ranges for fields in time intervals are limited by the
overall validity range of the TMG.

{\bf Priority:} Very Useful.
\end{reqlist}

\sreq{The \abbr should allow components to specify time instants that use 
the Gregorian calendar and UTC times.}
\begin{reqlist}
{\bf Priority:} Essential.
\end{reqlist}

\sreq{The \abbr should provide a mechanism for translating Gregorian calendar time instants to and from 
Julian days.}
\begin{reqlist}
{\bf Priority:} Useful.
\end{reqlist}

\sreq{The \abbr should provide support for a "generic", simple calendar.}
\begin{reqlist}
{\bf Background:}
Often it is useful to perform an idealized simulation with a calendar
that is an approximation to an actual calendar. A commonly used
configuration is a 360 day year containing 12 months of 30 days each.
Each day is exactly 86400 seconds. The period of rotation is
assumed to be exactly one day and the orbital period around the Sun
is assumed to be exactly one year.
These settings are a useful approximation for the Earth. However, for other
planets, or for idealised parameter space exploration experiments, the year 
length and day length need to be adjusted. The TMG generic calendar
should allow flexible specification of year and day lengths.

{\bf Priority:} Essential.
\end{reqlist}

\sreq{A component should only need to specify the {\bf YY MM DD H M S MS O} fields
that are relevant to that component.}
\begin{reqlist}
{\bf Background:}
If, for example, a component only needs to step forward in whole month intervals, then
it should be possible to specify \abbr time instants and time intervals in months only. 
The other fields should then default to appropriate values for the calendar.
{\bf Priority:} Very Useful.
\end{reqlist}

\sreq{The ability to synchorise or ``attach'' a clock to an external 
source should be supported.}
\begin{reqlist}
{\bf Background:} Forecast scenarios require latching key events to 
actual wall-clock time.

{\bf Priority:} Useful.
\end{reqlist}


\req{Continuation of \abbr state over ``multi-stage'' calculations is required.}
\begin{reqlist}
{\bf Background:} Clocks and alarms need to continue from their prior state when an
application is stopped and restarted.

{\bf Priority:} Very Useful.
\end{reqlist}

\req{Time resolutions and durations consistent with a very broad range
of Earth science related applications are required.}
\begin{reqlist}
{\bf Background:} Many components can be applied to a broad range of 
problems spanning time-scales of seconds to millions of years.

{\bf Priority:} Essential.
\end{reqlist}

\sreq{Time resolution of at less than or equalt to one millionth of a
second should be supported.}
\begin{reqlist}
{\bf Background:} Many component applications can be (and are) applied to 
problems involving small-scale fluid processes. These simulations may 
well employ time-steps of less than one second. Simulations that are 
performed in conjunction with laboratory experiments may involve sensor 
data, using sensors that can sample up to a rate of $10^6$ samples per 
second ( 1MS/s ).

{\bf Priority:} Essential.
\end{reqlist}

\sreq{Simulations with durations of at least 100,000 years, and 
preferably up to one-hundred million years should be supported.}
\begin{reqlist}
{\bf Background:} 
Hundred thousand year time scales arise very readily from studies involving
orbital variations that may influence glacial and inter-glacial transitions.
Ensembles of calculations that look at the impact
of plate configurations on climate can potentially span scenarios
covering millions of years (mantle convection occurs over many millions of
years). The \abbr could provide a useful, universal clock
to such simulations, provided it is designed to support a sufficiently
broad range of times. \footnote{
Note - a pair of 64-bit integers can represent, in seconds
and atto-seconds ($10^{-18}s$), a time range of about 300 biliion years!}

{\bf Priority:} Useful
\end{reqlist}

\req{Sufficient time instant and time interval accuracy is required.}
\sreq{Truly drift and truncation free behaviors should be supported.}
\begin{reqlist}
{\bf Background:}
An option must be available to create time intervals and time instants that
are free from truncation. An example of no drift behavior is as follows:

if adding (or subtracting) a time interval, $i_{1}$, to a time 
instant $\tau_{init}$ yields a new time instant $\tau_{final}$ i.e.

$\tau_{final}=\tau_{init}+i_1 $

then it must be possible to specify a fractional time 
interval $i_{2} = \frac{1}{2}\times i_{1}$ such that

$\tau_{final}=\tau_{init}+i_1\equiv\tau_{init}+i_2+i_2 $

and similarly for $i_{3} = \frac{1}{3}\times i_{1}$, $i_{n,m} = \frac{n}{m}\times i_{1}$.
Such behavior is not possible with finite-precision floating point arithmetic.



{\bf Priority:} Essential
\end{reqlist}

\sreq{Clock accuracy comparable with numerical scheme accuracy should be supported.}
\begin{reqlist}
{\bf Background:} Applications that employ an adaptive time-step for their numerical
procedures need to be able to drive their time manager clocks with that step.
This should be possible even for application time steps that are arbitrary 
floating point numbers i.e. not a whole number of seconds, or minutes.

{\bf Priority:} Essential
\end{reqlist}

\sreq{Applications should be able to query the actual values used in a clock or alarm.}
\begin{reqlist}
{\bf Priority:} Essential
\end{reqlist}

\req{Flexible support for manipulating clock and alarm settings is required}
\sreq{Time intervals should be able to increment and decrement time 
instants.}
\begin{reqlist}
{\bf Background:} Alarms will only be triggered when time is advanced.
Alarms are triggered when a time interval goes from being less than
the current time instant for the alarm to a time greater than or equal to the current
time instant for the alarm. It should be possible to request
a clock setting increment that will not trigger an alarm.

{\bf Priority:} Essential
\end{reqlist}
\sreq{Both once-only and periodic alarm settings should be supported}
\begin{reqlist}
{\bf Background:} Alarms can be set to happen at a single time
instant only or at a repeating time instant. Repeating time instants
should include every occurence of a particular day, on a certain
day of each month, at the start of a month, at the end of a month,
at the middle of a month.

{\bf Priority:} Essential
\end{reqlist}

\req{Time instant comparison support is required}
\sreq{Support for testing a pair of time instants for equality is required.}
\begin{reqlist}
{\bf Priority:} Essential
\end{reqlist}
\sreq{Support for calculating the time interval between a pair of time 
instants is required.}
\begin{reqlist}
{\bf Background:} Calculating the difference between
times with integer fraction representation in such a way that
the integer fractions are preserved is technically difficult
( well at least I can't think how to do it in general).
Time differences are only required to be returned as time intervals
that use finite precision. There is no requirement for
inifinite precision time intervals to be returned as
the result of a time difference calculation.

{\bf Priority:} Essential
\end{reqlist}

\req{Support for a variety of time output formats is required}
\begin{reqlist}
{\bf Background:} It is not possible to be define a single I/O format 
for textual output of time settings.
A facility to format times flexibly would be useful
for providing appropriate I/O, for example ``%x'' style 
or ``cout >>'' style format specifiers.

{\bf Priority:} Optional
\end{reqlist}

\def\refname{Reference Material}
\def\bibname{Reference Material}
\begin{thebibliography}{99}

\bibitem{1}
\textsl{International Earth Rotation Service (IERS)}
\begin{rawhtml}<A href=http://hpiers.obspm.fr>\end{rawhtml}
http://hpiers.obspm.fr \\
\begin{rawhtml}</A>\end{rawhtml}
\begin{rawhtml}<A href=http://hpiers.obspm.fr/eop-pc/products/bulletins.htm>\end{rawhtml}
IERS Bulletins \\
\begin{rawhtml}</A>\end{rawhtml}
\begin{rawhtml}<A href=http://hpiers.obspm.fr/eop-pc/models/constants.html>\end{rawhtml}
IERS Constants \\
\begin{rawhtml}</A>\end{rawhtml}

\bibitem{2}
\begin{rawhtml}<A href=//www.cgd.ucar.edu/cms/eaton/netcdf/NCAR-CSM.html>\end{rawhtml}
\textsl{CSM Time Conventions} 
\begin{rawhtml}</A>\end{rawhtml}

\bibitem{3}
\begin{rawhtml}<A href=http://www.omg.org>\end{rawhtml}
\textsl{OMG\\} 
\begin{rawhtml}</A>\end{rawhtml}
CORBA Time Service Specification Version 1.0
\begin{rawhtml}<A href=ftp://ftp.omg.org/pub/docs/formal/00-06-26.pdf>\end{rawhtml}
ftp://ftp.omg.org/pub/docs/formal/00-06-26.pdf
\begin{rawhtml}</A>\end{rawhtml}

\bibitem{4}
\begin{rawhtml}<A href=http://www.usno.navy.mil>\end{rawhtml}
US Naval Observatory\\
\begin{rawhtml}</A>\end{rawhtml}
\begin{rawhtml}<A href=http://tycho.usno.navy.mil>\end{rawhtml}
US Naval Observatory Time Pages\\
\begin{rawhtml}</A>\end{rawhtml}

\bibitem{5}
\begin{rawhtml}<A href=http://www.w3.org/TR/NOTE-datetime>\end{rawhtml}
W3C Date and Time Note
\begin{rawhtml}</A>\end{rawhtml}

\bibitem{6}
\begin{rawhtml}<A href=http://www.boulder.nist.gov/timefreq/general/faq.htm>\end{rawhtml}
NIST Time and Date Material
\begin{rawhtml}</A>\end{rawhtml}

\bibitem{6}
\begin{rawhtml}<A href=http://www.tondering.dk/claus/calendar.html>\end{rawhtml}
A Calendar FAQ.
\begin{rawhtml}</A>\end{rawhtml}

\bibitem{7}
\begin{rawhtml}<A href=http://www.mcs.vuw.ac.nz/technical/software/SGML/doc/iso8601/ISO8601.html>\end{rawhtml}
Some notes on the ISO8601 time and date specification standard.
\begin{rawhtml}</A>\end{rawhtml}

\end{thebibliography}

\end{document}
