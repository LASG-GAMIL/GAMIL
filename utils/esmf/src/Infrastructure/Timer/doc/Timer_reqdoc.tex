% $Id: Timer_reqdoc.tex,v 1.1.6.1 2002/04/24 03:25:57 erik Exp $
\documentclass[]{article}
\usepackage{html}
\usepackage[T1]{fontenc}

\begin{document}
\bodytext{BGCOLOR=white LINK=#083194 VLINK=#21004A}

\begin{titlepage}

\begin{latexonly}
\noindent {\bf Community Climate System Model} \\
\noindent National Center for Atmospheric Research, Boulder, CO \\
\vspace{2in}
\end{latexonly}

\begin{center}
{\Large\bf Timer Requirements} \\
\medskip
{\it Jim Rosinski, David Neckels }
\end{center}

\end{titlepage}

\tableofcontents

\newpage
%\section{Synopsis}
% $Id: Timer_syn.tex,v 1.1.6.1 2002/04/24 03:25:57 erik Exp $
\section{Overview}

This utility provides an API for timing code sections and
for general hardware profiling.


\section{Requirements}

\subsection{Timer toolkit}

\subsubsection{General Requirement}
A general purpose timer library will handle the timing of
code sections and reporting of the timing results.  The basic interface involve
a start method, which begins a named timer, a stop method, which stops the
timer, and a print which reports the results.

\subsubsection{Process Granularity}

The timer will have granularity to the process and thread level, supporting 
threads and MPI.  When results of a named timer are reported the
thread-id and process (MPI) will be reported in addition to the timer name.
Timers will be able to resolve the time spent in a single
thread and report this in a consistent manner on all platforms.

The thread support will be either for Pthreads or OMP threads.  These threads do
not work together, so the support will be for one, the other, or neither, and
will be set at compile or run time.

\subsubsection{Load balancing}

Where possible the library should provide the option to enable an absolute
timestamp at the start/stop pairs.  This will enable load balancing issues to
be more easily analyzed.

\subsection{Profiling Tools}

Hardware counters, when available, will be harnessed to provide unified and
portable profiling information.  Items such as cache hits/misses, MPI latency,
bandwidth and flop counts will be collected and reported where possible.
The calls to the profiling library will detect if the hardware counters are not
accessible and will not halt the program, but will merely become stub routines
and return immediately.   

Existing profilers such as PCL \cite{pcl} and PAPI \cite{papi} will either be used
or will influence the design of the library profiler.

\subsection{Compile Time Disable/Enable}

The library will contain compilation macros to allow all calls in the library
to be simple returns if the flags are turned off.  This will allow production
code to run without losing any speed loss incurred from the library.

There will be compilation flags to enable/disable:

\begin{itemize}
\item{Threads}
\item{MPI}
\item{Profiling (possibly PCL or PAPI)}
\end{itemize}

When these flags are off, all calls to these libraries will be compiled out and
the associated library will not need to be linked in, nor the headers
included. This allows users on systems not supporting these functions to use
the rest of the library, and it allows users who do not desire these
capabilities to disable them.

The library will also provide runtime flags such that if the library is
compiled in it may still be turned off or set to various levels of vigour via switches.  
These flags will be the same set as the compile set, with some additions.

\subsection{Implied Requirements}

The PCL or PAPI drivers will need to be installed on target platforms to enable profiling.

\subsection{Restrictions}

On some platforms the hardware counters are not available to the granularity specified
above, so the profiler will only be able to resolve to the level supported on each
platform.

\section{Review Status}

\noindent {\bf Requirements Review} \\

\begin{tabular}{r p{1.3in} p{2in}}
{\bf Review Date:} & <Date> \\ \\
{\bf Reviewers:}   & Cecelia Deluca     & NCAR \\
                   & Brian Kauffman     & NCAR \\
                   & Tony Craig         & NCAR \\
                   & Erik Kluzek        & NCAR \\
                   & Nancy Norton       & NCAR \\
                   & Keith Lindsay      & NCAR \\
                   & Steve Gombosi      & NCAR \\
                   & Jim Rosinski       & NCAR
\end{tabular}
%\section{Glossary}
%% $Id: Timer_glos.tex,v 1.1.6.1 2002/04/24 03:25:56 erik Exp $
\section{Glossary}

\begin{description}

\item [STDOUT] \label{glos:STDOUT} In fortran this is the {\tt unit=6}.  In {\tt C} this is where output
using a non-qualified {\tt printf} goes.

\end{description}













%\section{Bibliography}
\bibliography{Timer} 
\bibliographystyle{plain}
\addcontentsline{toc}{section}{Bibliography}

\end{document}

