\subsubsection{General Use}

Any number of {\tt ESMF\_Log's} may be created using {\tt ESMF\_LogNew}.  Each
log will have a unique file-name stem, which is passed to this call.  In the event that
the process is not using MPI or the option for a single output file is selected, all
output will go directly to a file with the stem name.  Otherwise one file will be
created for each MPI process with the stem followed by the process number as the 
output file name.



\subsubsection{Example Program}

The following program creates several {\tt ESMF\_Log's} and uses them to
output various messages.

\begin{verbatim}
        program LogEx
        Use ESMF_AppMod
        Use ESMF_LogMod
        implicit none

        integer :: ivar, i, ierr
        real(8) :: dvar 
        real(4) :: fvar 
        type(ESMF_Log) :: log, testlog
        type(ESMF_App) :: app

! Create the application.  Every ESMF program must create an app.
        app = ESMF_AppNew()
! Create a new log with stem "logfile".
        log = ESMF_LogNew("logfile", ESMF_LOGSTATE_VERBOSE, 1)
! Create another log with stem "testlog".
        testlog = ESMF_LogNew("testlog",
     &  ESMF_LOGSTATE_VERBOSE, 0)

! Test printing a string.
        call ESMF_LogPrint(testlog, ESMF_LOGLEVEL_INFO,
     &  "AAA: Hello Logfile")

! Set some various types of variables with values.
        ivar = 2
        dvar = 3.1459
        fvar = 0.145987

! Demonstrate printing the variables in different formats.
        call ESMF_LogPrint(testlog, ESMF_LOGLEVEL_ERROR,
     &  "Assorted Vars:%s%08d\ndvar=%.2lf,fvar=%.4f\n",
     &  "i=", ivar, dvar, fvar)

! Flush the testlog output.
        call ESMF_LogFlush(testlog)
        
! Demonstrate the use of the log in a threaded section.
!$OMP PARALLEL DO PRIVATE (i)
        do i=1, 10 
                call ESMF_LogPrint(log, ESMF_LOGLEVEL_INFO,
     &  "Hello from do loop, iter= %d\n", i)

        call ESMF_LogPrint(testlog, ESMF_LOGLEVEL_INFO, 
     &  "Called from OMP Parallel do\n")
        enddo
!$OMP END PARALLEL DO

! Delete the log when finished.  This closes output files and does
! the necessary cleanup.
        call ESMF_LogDelete(testlog)
        call ESMF_LogDelete(log)

! Each app must delete the app handle.  This finializes the program.
        call ESMF_AppDelete(app)

        end program 
\end{verbatim}

\subsubsection{Output Matrix}

The following matrix determines which messages are output in a given log state.
\smallskip
\noindent
\begin{tabular}{|p{1.5in}p{1.0in}p{1.0in}p{1.0in}|} \hline
                  &| {\bf Log Level} & & \\ \hline
{\bf Log State}   &| {\bf INFO }& {\bf ERROR } & {\bf TIMER} \\ \hline
{\bf QUIET}       &| off & off & off \\ \hline
{\bf NORMAL}      &| on & on & off \\ \hline
{\bf TIMER}       &| off & off & on \\ \hline
{\bf VERBOSE}     &| on & on  & on  \\ \hline
\end{tabular}

\subsubsection{Example Program Output}

\noindent
{\bf Output from testlog}
\begin{verbatim}
>>Time=0000.00662,PID=0,TID=0,ERROR  :Assorted Vars:i=0000000
>>Time=0000.00662,PID=0,TID=0,ERROR  :dvar=3.15,fvar=0.1460
>>Time=0000.09440,PID=0,TID=1,INFO   :Called from OMP Parallel do
>>Time=0000.09487,PID=0,TID=2,INFO   :Called from OMP Parallel do
\end{verbatim}

\noindent
{\bf Output from logfile}
\begin{verbatim}
>>Time=0000.09349,PID=0,TID=0,INFO   :Hello from do loop, iter= 1
>>Time=0000.09421,PID=0,TID=3,INFO   :Hello from do loop, iter= 9
\end{verbatim}

Each line of output in the log file is prepended with a descriptor line.  The time,
process Id, thread Id, and logging level appear before each 
line of output.  This allows for unix tools like {\tt grep} to be used to 
post-process the output file.  Sorting several files by the time field allows the user
to get a single file which shows the relative flow of events.  If
the output is cut after the ``:'', the output appears exactly as it would if a simple
{\tt write} or {\tt printf} statement were used.

