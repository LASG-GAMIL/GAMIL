% $Id: Timer_def.tex,v 1.1.6.1 2002/04/24 03:25:55 erik Exp $
The timing library and profiling library will be implemented as a single unit.  
\subsubsection{Subroutines}

\begin{itemize}
\item{\bf MFM\_TimerInit}
Configures and sets options for the timer.  Establishes which types of
profiling and timing statistics are to be collected.  Also initializes any
internal data structures needed for the library.  If the {\tt name} field is 
set to {\tt MFM\_ALL\_TIMERS}, the basic timer options are set.  Any timer that
is created after this call will utilize the options set with {\tt
MFM\_ALL\_TIMERS} unless a separate call to {\tt MFM\_TimerInit} is made with
that timer name.  If special behavior is desired for a given timer, the init
function may be called with the name of the timer as the first argument.  When
this specific name is
given, the options specified apply to timers of that name only (If the timer is
used in more than one thread there can be multiples with the same name).  This 
function Should be called in a non-threaded code section.

\begin{verbatim}
        subroutine MFM_TimerInit(name, option1, value1, option2, value2, ...)
        character*(*) :: name            ! Timer name or "MFM_ALL_TIMERS" for all timers
        integer :: option1               ! The option to set
        integer :: value2                ! The value of option1
        ! etc...
\end{verbatim}

Option values are either one (enable) or zero (disable).  The possible options
are:
\begin{itemize}
\item{\tt MFM\_STUB\_TIMERS} Causes all calls to timer functions to return immediately.
\item{\tt MFM\_USRSYS} (Measure) User and system time.
\item{\tt MFM\_WALL} Wall clock times.
\item{\tt MFM\_LD1CACHE\_MISS} Level 1 cache misses.
\item{\tt MFM\_LD2CACHE\_MISS} Level 2 cache misses.
\item{\tt MFM\_CYCLES} Clock cycles in this process/thread.
\item{\tt MFM\_ELAPSED} Elapsed clock cycles (all programs).
\item{\tt MFM\_FP\_INSTR} Floating point instructions.
\item{\tt MFM\_LOADSTORE\_INSTR} Load and store instructions.
\item{\tt MFM\_INSTR} Instruction count.
\item{\tt MFM\_STALL} Sum of cycles the process/thread is stalled.

\end{itemize}

\item{\bf MFM\_TimerStart}

This function makes a timer/profiler start counting.

\begin{verbatim}
        subroutine MFM_TimerStart(name)
        character(*) :: name        ! The timer name.
\end{verbatim}

Each timing event will be done via the start/stop pairs.  A timer may be called
multiple times, but within a given thread, the {\tt MFM\_TimerStart} routine
must not be called twice without an intervening call to {\tt MFM\_TimerStop}.
If a start call is made before a start, the timer will print an error
message and ignore the particular timing event.  The timer will continue to operate
if called again in a correct sequence.
An internal list of named  timers is kept, so repeated calls find the original
timer and re-use it.  These timers will keep averages, minimums and maximums,
and this will be printed in calls to {\tt MFM\_Timer\_Print}.

\item{\bf MFM\_TimerStop}

Stops a named timer.
\begin{verbatim}
        subroutine MFM_TimerStop(name)
        character*(*) :: name        ! The timer name.
\end{verbatim}

\item{\bf MFM\_TimerReset}

Resets the given timer.  Zeros all accumulated times.

\begin{verbatim}
        subroutine MFM_TimerReset(name)
        character *(*):: name
\end{verbatim}

\item{\bf MFM\_TimerPrint}

Prints the results of the timer using the given {\tt ESMF\_Log}.  The name should be
from a previous stop/start pair or {\tt MFM\_ALL\_TIMERS} for all timers.  The routine prints
the timer to the specified logging channel.

\begin{verbatim}
        subroutine MFM_TimerPrint(name, log)
        character *(*):: name
        MFM_LogHandle :: log
\end{verbatim}

\item{\bf MFM\_TimerStamp}

Returns a time stamp as an 8 byte float.  

\begin{verbatim}
        subroutine MFM_TimerStamp(stamp)
        real(8) :: stamp
\end{verbatim}


\end{itemize}
