% $Id: TimeMgmt_desdoc.tex,v 1.1.6.1 2002/04/24 03:25:48 erik Exp $

\documentclass[]{article}

\usepackage{epsf}
\usepackage{html}
\usepackage[T1]{fontenc}

\textwidth 6.5in
\textheight 8.5in
\addtolength{\oddsidemargin}{-.75in}

\begin{document}

\bodytext{BGCOLOR=white LINK=#083194 VLINK=#21004A}

\begin{titlepage}

\begin{center}
{\Large Earth System Modeling Framework } \\
\vspace{.25in}
{\Large {\bf Time Management Library Design}} \\
\vspace{.25in}
{\large {\it Authors}}
\vspace{.5in}
\end{center}

\begin{latexonly}
\vspace{5.5in}
\begin{tabular}{p{5in}p{.9in}}
\hrulefill \\
\noindent {\bf NASA High Performance Computing and Communications Program} \\
\noindent Earth and Space Sciences Project \\
\noindent CAN 00-OES-01 \\
\noindent http://www.esmf.ucar.edu \\
\end{tabular}
\end{latexonly}

\end{titlepage}

\tableofcontents

\newpage
%\section{Synopsis}
% $Id: TimeMgmt_syn.tex,v 1.1.6.1 2002/04/24 03:25:50 erik Exp $
\section{Synopsis}

The Earth System Modeling Framework (ESMF) Time Management Library provides utility functions for time and date calculations, and higher-level functions that control model time stepping and alarms.  










The interface has been designed with use by Earth system models specifically in mind. The use of encapsulated classes for date, time and time of day  
allows for a wide range of values and precision so that applications from weather forecasting to paleoclimate 
simulation are supported.  

The design of this Time Management Library is based on the time management 
utilities in the \htmladdnormallink{Flexible Modeling System}
{http://www.gfdl.gov/~fms} (FMS) from the NOAA Geophysical 
Fluid Dynamics Laboratory.

It will be built on infrastructure derived from the \htmladdnormallink
{Portable Extensible Toolkit for Scientific Computation}
{http://www-fp.mcs.anl.gov/petsc/} (PETSc)

%\section{Algorithmic Description}
%% $Id: comp_alg.tex,v 1.1.6.1 2002/04/24 03:25:25 erik Exp $

\section{Algorithmic Description}

<Description of the continuous and discrete scientific algorithms used
in the software.  May reference rather than describe algorithms explicitly.>














%\section{Requirements}
% $Id: TimeMgmt_req.tex,v 1.1.6.1 2002/04/24 03:25:49 erik Exp $

\section{Requirements}

\subsection{Date and Time Utility Requirements}

\begin{itemize}

\item Support \htmlref{no-leap}{glos:noleap} and Gregorian calendars.

\item Provide the following calculations:

\begin{itemize}

\item Given a date, compute a new date that is either earlier or later by a specified time interval.

\item Given two values of date compute the time interval between them.

\item Verify whether or not one date is later than another.

\item Compute the \htmlref{day of year}{glos:dayofyear} given date.

\end{itemize}

\item Handle a variety of representations of time intervals.  The initial requirement is supporting a 
precision of 1 second.  Future requirements include handling millisecond discretizations and floating 
point representations of time intervals.

\item Handle time intervals with a range of 20,000 years.
 
\item Date to string conversion.
\end{itemize}

\subsection{Time Manager Requirements}
\begin{itemize}

\item User will specify timestep size, start and stop dates. User may optionally specify the base date. 
By default the base date equals the start date.

\item Support changing the timestep size during a simulation. 

\item Provide functions to:

\begin{itemize}
\item Query timestep size, start and base dates.

\item Query current timestep number and properties of the current timestep such as date, 
time, and day of year at the endpoint of the current timestep.

\item Convert between time and date.
\end{itemize}

\item Provide alarm functions:

\begin{itemize}
\item Alarms that go off periodically can be specified by a period and an offset.

\item Alarms that go off when year and month boundaries are crossed.

\item A component or parameterization queries whether its alarm is on or off.
\end{itemize}

\item Must be able to operate in a "restart" mode.

\end{itemize}









%\section{Architecture}
%% $Id: comp_arch.tex,v 1.1.6.1 2002/04/24 03:25:26 erik Exp $

\section{Architecture}

<Describe layering strategy and interaction of major components,
provide examples of high-level interfaces.>










\section{Time Class}

%\subsection{Description}
% $Id: Time_desc.tex,v 1.1.6.1 2002/04/24 03:25:50 erik Exp $

The time class is used to perform fundamental operations related to time 
intervals.  A time consists of a number of days and a time of day, which
is contained in a separate time of day class.  Intervals less than a day 
may be represented by integer seconds, integer seconds plus milliseconds, 
or float seconds.  Currently only integer seconds are implemented.

Basic operations supported by the time class include setting and getting
time attributes, copying times, and incrementing and decrementing times.











\subsection{Design}

\subsubsection{Class Definition}

The attributes of the time class expressed as a Fortran 90 derived type are:

\noindent type time\_t \\
\indent private \\
\indent integer :: days \\
\indent integer :: seconds \\
\noindent end type time\_t

\subsubsection{Design Strategy}

The precision and range requirements of time intervals and dates determine their representation in 
terms of native machine types (i.e, floating point numbers and integers). The time and date classes
are defined to represent time intervals and dates, respectively, in order to insulate the user interface 
from the underlying representation.  This not only makes it easier to pass arguments (e.g., passing a
date argument versus passing all the integers that represent the components of a date), it also means 
that the interface doesn't have to change if the underlying representation were changed to support 
new precision or range requirements.

When an error is encountered the functions in this library will call a private error handling routine 
which issues a message explaining the error and then calls exit. The user may be able to customize the
error handling.

\section{Date Class}

%\subsection{Description}
% $Id: Date_desc.tex,v 1.1.6.1 2002/04/24 03:25:48 erik Exp $

The {\tt Date} class provides a set of functions for manipulating dates.
These include setting and retrieving dates, incrementing and decrementing 
dates by a specified time interval, taking the difference of two dates,
determining whether one date is later than another, and computing the
day of year of a given date.
   
The {\tt Date} class contains attributes representing year, month and day 
quantities and a time of day.  It also contains a calendar which 
stores, for a given year, such quantities as the number of days per 
month and per year.  Gregorian and no-leap year calendars are currently 
supported.  

The algorithm to convert from Gregorian to Julian days is from 
Henry F. Fliegel and Thomas C. Van Flandern, in Communications of 
the ACM (CACM, volume 11, number 10, October 1968, p.657).  Julian 
day refers to the number of days since a reference day.  For the 
algorithm used, this reference day is November 24, -4713 in the Gregorian 
calendar.  The algorithm is valid through all future dates, assuming 
standard corrections are applied (at 4 years, 100 years,
and 400 years).




\subsection{Design}

\subsubsection{Class Definition}

The attributes of the date class expressed as a Fortran 90 derived type are:

\noindent type date\_t \\
\indent private \\
\indent integer :: year \\
\indent integer :: month \\
\indent integer :: day \\
\indent integer :: sec \\
\noindent end type date\_t

\subsubsection{Design Strategy}

As with the time class, the date class hides the representation of its internal attributes from the 
user, thereby increasing the class's extensibility and portability.

\section{Review Status}

\noindent{\bf Requirements and Design Review} \\

\begin{tabular}{r p{1.3in} p{2in}}
{\bf Review Date:} & March 1, 2001 \\ \\
{\bf Reviewers:}   & Byron Boville        & NCAR/CGD \\
                   & Dave Williamson      & NCAR/CGD \\
                   & Phil Rasch           & NCAR/CGD \\
                   & Cecelia DeLuca       & NCAR/SCD \\
                   & Jim Rosinski         & NCAR/CGD
\end{tabular}

%\section{Glossary}
% $Id: TimeMgmt_glos.tex,v 1.1.6.1 2002/04/24 03:25:49 erik Exp $
\section{Glossary}

\begin{description}

\item [date] \label{glos:date} A date is used to specify an instant of time at the Greenwich meridian.  It consists
of year, month, day of month and time of day components.

\item [time] \label{glos:time} A time is used to specify an interval of time. 
              
\item [day of year] \label{glos:dayofyear} The day number in the calendar year. January 1 is day 1 of the year. 
Day of year expressed in a floating point format is used to express the day number plus the time of day 
at Greenwich. For example, assuming a Gregorian calendar: 

\begin{tabular}{ll}
{\bf date}              & {\bf day of year} \\
\hline 
10 January 2000, 6Z     & 10.25 \\
31 December 2000, 18Z   & 366.75 
\end{tabular}

\item [no-leap calendar] \label{glos:noleap} Every year uses the same months and days per month as in a non-leap 
year of a Gregorian calendar.

\end{description}













%\section{Bibliography}
%\bibliography{comp} 
%\bibliographystyle{plain}
%\addcontentsline{toc}{section}{Bibliography}

\section*{Appendix:  Fortran Interface}
\addcontentsline{toc}{section}{Appendix:  Fortran Interface}

%\section{ESMF_Time Interface}
\input{ESMF_TimeMod}

%\section{ESMF_Date Interface}
%\input{ESMF_Date}

%\section{ESMF_TimeMgr Interface}
%\input{ESMF_TimeMgr}

%\section{ESMF_Alarm Interface}
%\input{ESMF_Alarm}

%\section*{Appendix:  C Interface}
%\addcontentsline{toc}{section}{Appendix:  C Interface}

%\section{MF_Time Interface}
%\input{MF_Time}

%\section{MF_Date Interface}
%\input{MF_Date}

%\section{MF_TimeMgr Interface}
%\input{MF_TimeMgr}

%\section{MF_Alarm Interface}
%\input{MF_Alarm}

\end{document}
A







