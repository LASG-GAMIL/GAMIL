% $Id: Log_reqdoc.tex,v 1.1.6.1 2002/04/24 03:25:36 erik Exp $
\documentclass[]{article}
\usepackage{html}
\usepackage[T1]{fontenc}

\begin{document}
\bodytext{BGCOLOR=white LINK=#083194 VLINK=#21004A}

\begin{titlepage}

\begin{latexonly}
\noindent {\bf Community Climate System Model} \\
\noindent National Center for Atmospheric Research, Boulder, CO \\
\vspace{2in}
\end{latexonly}

\begin{center}
{\Large\bf Log Requirements} \\
\medskip
{\it David Neckels }
\end{center}

\end{titlepage}

\tableofcontents

\newpage
%\section{Synopsis}
% $Id%
\section{Overview}

This utility provides an organized method for writing diagnostic
output.  In a large program
writing to {\tt STDOUT, STDERR} often results in a mass of output that
is either too cluttered to understand or that is lost due to pipe 
re-direction.  The log provides a message output channel that organizes
its output by the author, the time of writing, and the class of the 
message.  This opens the way for standard unix tools like {\tt grep, sort, cut}
and others to manipulate the output files (the output is guarenteed to show
up in a pre-designated place) to sort through the results and retrieve
the desired information.


\section{Requirements}

\subsection{Interface}

An interface for logging will be created that will be geared toward 
simplicity of use.  
The logging API will be as similar to the existing fortran {\tt write} as
possible, and will require minimal change of existing code to use.

\subsection{Logging States/Levels}
Each call to the logger will contain a logging
level.  The program itself will have a logging state, which may be set by an
environment variable, and possibly adjusted at runtime via a signal.
A matrix will be created that enumerates which logging levels are output during 
which state.  Calls to logging in an inactive state will return immediately.

\subsection{Process Organization}

The library will allow output from different MPI processes to be distinguished
or grouped.  Output from a given unit 
can be either output all to separate files or output all to the same
file with distinguishing tags prepended to the output.  This will allow tools
like {\tt grep} to operate on the file and isolate output from any given unit.

The output will be unix friendly, i.e. each line of output will be prefaced
with information that will allow tools like grep, sort, etc.. to be used to
organize the results.

\subsection{Log Flush}

There will be a flush call so that output will be available in the output arena
once this call has been made.
\section{Review Status}

\noindent {\bf Requirements Review} \\

\begin{tabular}{r p{1.3in} p{2in}}
{\bf Review Date:} & <Date> \\ \\
{\bf Reviewers:}   & Cecelia Deluca     & NCAR \\
                   & Brian Kauffman     & NCAR \\
                   & Tony Craig         & NCAR \\
                   & Erik Kluzek        & NCAR \\
                   & Nancy Norton       & NCAR \\
                   & Keith Lindsay      & NCAR \\
                   & Steve Gombosi      & NCAR \\
                   & Jim Rosinski       & NCAR
\end{tabular}
%\section{Glossary}
%% $Id: Log_glos.tex,v 1.1.6.1 2002/04/24 03:25:36 erik Exp $
\section{Glossary}

\begin{description}

\item [STDOUT] \label{glos:STDOUT} In fortran this is the {\tt unit=6}.  In {\tt C} this is where output
using a non-qualified {\tt printf} goes.

\end{description}













%\section{Bibliography}
\bibliography{Log} 
\bibliographystyle{plain}
\addcontentsline{toc}{section}{Bibliography}

\end{document}






